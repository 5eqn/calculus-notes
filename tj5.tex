\documentclass[UTF8,a4paper,11pt]{ctexart}
\usepackage{listings} 
\usepackage{xcolor} 
\lstset{
    basicstyle=\tt,
    keywordstyle=\color{purple}\bfseries,
    identifierstyle=\color{brown!80!black},
    commentstyle=\color{gray}
    showstringspaces=false,
    numbers = left,                
    numberstyle = \small,               
}
\title{同济高数第五章笔记}
\author{5eqn}
\date{\today}
\begin{document}
  \maketitle
  \section{定义}
    定积分按照求面积极限定义。
  \section{逼近}
    有几种逼近方法:
    \subsection{矩形法}
      不做线性插值。
    \subsection{梯形法}
      有线性插值,
      体现为削弱两头。
    \subsection{辛普森法}
      也叫抛物线法,每两格做一次抛物线插值。
      不仅削弱两头,而且奇数(抛物线中间)权值更高。
  \section{性质}
    \subsection{线性}
      包括单位元、可常量乘和可加性。
    \subsection{有界判定}
      被积函数有界,就可以构造出定积分结果的上界和下界。
      只要能框住被积函数,积分结果也可以框住。
    \subsection{中值定理}
      用有界判定放缩之后,
      加上连续函数介值定理即可。
  \section{积分上限函数}
    就是从一个原函数 $f$ 利用积分
    构造出一个求导后等于原函数的函数的方法。
    \[
    F(x)=\int_{a}^{x} f(t)\mathrm{d}t 
    .\] 
    要对 $\int_{a}^{b} f(x)\mathrm{d}x $ 求值,
    使用牛顿-莱布尼兹公式找到原函数的 $C=F(a)$ 即可。
  \section{求值}
    注意按照实际变量的值域改积分符号上的值域即可。
  \section{反常积分}
    把反常的地方通过极限变成正常积分即可。
    \subsection{定义域无穷}
      对无界方向的定义域取极限。
    \subsection{值域无界}
      对瑕点取单向极限,
      本来要积到瑕点,
      现在只需要积到瑕点附近即可。
      注意可能要拆成多个积分段。
    \subsection{审敛}
      直接和反比例函数 $\frac{N}{x}$ 对比即可,
      注意 $N$ 可以任意小。
      注意如果要判断收敛,即使等于也不够,
      因为积分出来是对数数量级,
      所以要使用极限思想找到 $p > 1$ 的幂次,
      使得 $\frac{N}{x^{p}}$ 满足要求。
      对无界函数,位移一下反比例函数即可。
      
\end{document}
