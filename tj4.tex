\documentclass[UTF8,a4paper,11pt]{ctexart}
\usepackage[margin=1in,xetex]{geometry}
\usepackage{listings} 
\usepackage{xcolor} 
\usepackage{amsmath}
\usepackage{amssymb}
\newtheorem{definition}{定义}
\newtheorem{theorem}{定理}
\newtheorem{proof}{证明}
\newtheorem{lemma}{引理}
\lstset{
  basicstyle=\tt,
  keywordstyle=\color{purple}\bfseries,
  identifierstyle=\color{brown!80!black},
  commentstyle=\color{gray},
  showstringspaces=false,
  numbers=left,                
  numberstyle=\small,               
}
\title{不定积分笔记}
\author{5eqn}
\date{\today}
\begin{document}
  \maketitle
  \begin{definition}
    不定积分是求导的逆运算.
  \end{definition}
  \begin{theorem}
    不定积分满足线性.
  \end{theorem}
  \section{不太有迹可循的积分路径}
    \subsection{分式}
      根据$\left(\tan x\right)^{\prime}=1+\tan ^{2}x$易得
      \[
      \begin{aligned}
        \int \frac{\mathrm{d}x}{1+x^{2}}&=\arctan x +C.
      \end{aligned}
      \]

      根据$\left(\sin x\right)^{\prime}=\sqrt{1-\sin ^{2}x}$易得
      \[
      \begin{aligned}
        \int \frac{\mathrm{d}x}{\sqrt{1-x^{2}}}&=\arcsin x+C.
      \end{aligned}
      \]

      运用拆分思想得到
      \[
      \begin{aligned}
        \int \frac{1}{x^{2}-a^{2}}\mathrm{d}x&=
        \frac{1}{2a}\ln \left| \frac{x-a}{x+a} \right|+C.
      \end{aligned}
      \]

      换元使得$x=a \sin t$, 可以求得
      \[
      \begin{aligned}
        \int \sqrt{a^{2}-x^{2}}\mathrm{d}x&=
        \frac{a^{2}}{2}\arcsin \frac{x}{a}+\frac{1}{2}x\sqrt{a^{2}-xs^{2}}+C.
      \end{aligned}
      \]
      
      换元使得$x=a \tan t$或$x = a \sec t$, 可以求得
      \[
      \begin{aligned}
        \int \frac{\mathrm{d}x}{\sqrt{x^{2}+a^{2}}}&=
        \int \sec t \mathrm{d}t
        \\&=\ln \left| \sec t + \tan t \right|+C
        \\&=\ln \left(x+\sqrt{x^{2}+a^{2}}\right)+C_1,\\
        \int \frac{\mathrm{d}x}{\sqrt{x^{2}-a^{2}}}\left(x>a\right)&=
        \int \sec t \mathrm{d}t
        \\&=\ln \left(\sec t+\tan t\right)+C
        \\&=\ln \left(x+\sqrt{x^{2}-a^{2}}\right)+C_1,\\
        \int \frac{x^{3}}{\left(x^{2}-2x+2\right)^{2}}\mathrm{d}x&=
        \int \frac{\left(\tan t+1\right)^{3}}{\sec ^{4}t}\cdot \sec ^{2}t \mathrm{d}t
        \\&=\int \left(\sin ^{3}t \cos ^{-1}t+3\sin ^{2}t+3\sin t \cos t+\cos ^{2}t\right)\mathrm{d}t
        \\&=-\ln \cos t-\cos ^{2}t+2t-\sin t \cos t+C.
      \end{aligned}
      \]
      
      将复杂有理分式拆成多项式, 
      $\frac{P_1\left(x\right)}{\left(x-a\right)^{k}}$
      和$\frac{P_2\left(x\right)}{\left(x^{2}+px+q\right)^{l}}$, 
      即可采用上面的方法求出.

      对于有根式的情况,
      进行一个特别小以至于能去掉所有根式的换元即可,
      例如题目中同时含有$\sqrt[3]{x}$和$\sqrt{x}$,
      那么进行换元$x=u^{6}$即可.

    \subsection{三角}
      根据三角函数求导规则, 易得
      \[
      \begin{aligned}
        \int \cos x \mathrm{d}x&=\sin x+C,\\
        \int \sin x \mathrm{d}x&=-\cos x+C,\\
        \int \sec ^{2}x \mathrm{d}x&=\tan x+C,\\
        \int \csc ^{2}x \mathrm{d}x&=-\cot x+C,\\
        \int \sec x \tan x \mathrm{d}x&=\sec x+C,\\
        \int \csc x \cot x \mathrm{d}x&=-\csc x+C.
      \end{aligned}
      \]

      换元掉$\sin x$或$\cos x$, 能够求出$\int \sin ^{2k+1}x \cos^{n} x$
      或$\int \sin ^{n}x \cos ^{2k+1}x$, 以及
      \[
      \begin{aligned}
        \int \tan x \mathrm{d}x&=-\ln \left| \cos x \right|+C,\\
        \int \cot x \mathrm{d}x&=\ln \left| \sin x \right|+C.
      \end{aligned}
      \]

      对于$\sin ^{2k}x \cos ^{2t}x$, 可以降幂后通过前面的方法求出.

      换元掉$\tan x$或$\sec x$, 能够求出$\int \tan ^{n}x \sec ^{2k}x$
      或$\int \tan ^{2k-1}x \sec ^{n}x$, 以及
      \[
      \begin{aligned}
        \int \csc x \mathrm{d}x&=\ln \left| \tan \frac{x}{2} \right|+C
        \\&=\ln \left| \csc x -\cot x \right|+C,\\
        \int \sec x \mathrm{d}x&=\ln \left| \sec x + \tan x \right|+C.
      \end{aligned}
      \]

      对于复杂的三角函数组合, 可以利用
      \[
      \begin{aligned}
        \begin{cases}
          \sin x=\frac{2\tan \frac{x}{2}}{1+\tan ^{2}\frac{x}{2}},\\
          \cos x=\frac{1-\tan ^{2}\frac{x}{2}}{1+\tan ^{2}\frac{x}{2}}
        \end{cases}
      \end{aligned}
      \]
      进行换元$u=\tan \frac{x}{2}$, 
      即可转化成有理分式的积分.

    \subsection{混合}
      分部积分具备积一边导一边的含义, 即:
      \[
      \begin{aligned}
        \int uv \mathrm{d}x=u \int v \mathrm{d}x-\iint v \mathrm{d}x \mathrm{d}u.
      \end{aligned}
      \]
      
      因此, 只要有积分不影响整体性质的指数型变量,
      另外一边求导后能变得更简单, 
      都可以不断导其它部分, 例如:
      \[
      \begin{aligned}
        \int x \cos x \mathrm{d}x&=x \sin x-\int \sin x \mathrm{d}x
        \\&=x \sin x+\cos x+C,\\
        \int x e^{x}\mathrm{d}x&=xe^{x}-\int e^{x}\mathrm{d}x
        \\&=e^{x}\left(x-1\right)+C.
      \end{aligned}
      \]
      
      只要有求导后能明显让整体变得简单的变量,
      另外一边积分后也不会让整体变得复杂,
      就可以开导, 例如:
      \[
      \begin{aligned}
        \int x \ln x \mathrm{d}x&=\frac{x^{2}}{2}\ln x-\int \frac{x^{2}}{2}\mathrm{d}\left(\ln x\right)
        \\&=\frac{x^{2}}{2}\ln x-\frac{x^{2}}{4}+C,\\
        \int \arccos x \mathrm{d}x&=x \arccos x-\int x \mathrm{d}\left(\arccos x\right)
        \\&=x \arccos x-\sqrt{1-x^{2}}+C.
      \end{aligned}
      \]

      如果两边求导积分后可能同构,
      可以利用这个性质列出方程后解出答案, 例如:
      \[
      \begin{aligned}
        \int e^{x}\sin x \mathrm{d}x&=e^{x}\sin x-\int e^{x}\cos x \mathrm{d}x
        \\&=e^{x}\sin x-e^{x}\cos x-\int e^{x}\sin x \mathrm{d}x,\\
        \int \sec ^{3}x \mathrm{d}x&=\sec x \tan x-\int \sec x \tan ^{2}x \mathrm{d}x
        \\&=\sec x \tan x+\ln \left| \sec x+\tan x \right|-\int \sec ^{3}x \mathrm{d}x,
      \end{aligned}
      \]
      解得
      \[
      \begin{aligned}
        \int e^{x}\sin x \mathrm{d}x&=\frac{1}{2}e^{x}\left(\sin x-\cos x\right)+C,\\
        \int \sec ^{3}x \mathrm{d}x&=\frac{1}{2}\left(\sec x \tan x+\ln \left| \sec x + \tan x \right|\right)+C.
      \end{aligned}
      \]
      
\end{document}
