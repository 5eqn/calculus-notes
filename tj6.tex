\documentclass[UTF8,a4paper,11pt]{ctexart}
\usepackage{listings} 
\usepackage{xcolor} 
\usepackage{amsmath}
\lstset{
  basicstyle=\tt,
  keywordstyle=\color{purple}\bfseries,
  identifierstyle=\color{brown!80!black},
  commentstyle=\color{gray},
  showstringspaces=false,
  numbers=left,                
  numberstyle=\small,               
}
\title{高数第六章笔记}
\author{5eqn}
\date{\today}
\begin{document}
  \maketitle
  \section{面积}
    \subsection{直角坐标}
      直接用定积分定义积分即可.
      例如对于抛物线$y=x^{2}, x \in [-1,1]$,
      可以按照如下方式计算:
      \[
      \begin{aligned}
        S&=\int_{-1}^{1}x^{2}\mathrm{d}x
        \\&=\frac{2}{3}
      .\end{aligned}
      \]
    \subsection{参数方程}
      先按照定积分定义写出式子,
      然后进行换元即可.
      例如对于椭圆$\frac{x^{2}}{a^{2}}+\frac{y^{2}}{b^{2}}=1$的第一象限,
      我们有参数方程
      \[
      \begin{aligned}
        \begin{cases}
          x=a \cos t ,\\
          y=b \sin t
        \end{cases}
      .\end{aligned}
      \]
      
      因此可以这样计算:
      \[
      \begin{aligned}
        S&=\int_{0}^{a}y\mathrm{d}x
        \\&=\int_{\frac{\pi}{2}}^{0}b\sin t \left(-a \sin t\right) \mathrm{d}t
        \\&=ab\int_{0}^{\frac{\pi}{2}}\sin ^{2}t \mathrm{d}t
        \\&=\frac{\pi}{4}ab
      .\end{aligned}
      \]
    \subsection{极坐标}
      仿照定积分的思想,
      每个角度对应一个扇形,
      扇形的面积是$\frac{\rho^{2}\left(\theta\right)}{2}\mathrm{d}\theta$.
      例如对于阿基米德螺线$\rho=a\theta$的一圈,
      可以这样计算:
      \[
      \begin{aligned}
        S&=\int_{0}^{2\pi}\frac{a^{2}\theta^{2}}{2}\mathrm{d}\theta 
        \\&=\frac{4}{3}a^{2}\pi^{3}
      .\end{aligned}
      \]

      选取不同的单元作为元素本质上是将不同的空间映射到直角坐标空间,
      对于高维积分最终也需要映射到二维.

  \section{体积}
    \subsection{旋转体}
      对于旋转体, 每个函数值会旋转出一个圆,
      面积是$\pi f^2\left(x\right)$,
      因此可以将体积映射到直角坐标的面积.
      例如对于旋转椭球体右半部分,
      可以这样计算:
      \[
      \begin{aligned}
        V&=\int_{0}^{a}\pi y^{2}\mathrm{d}x
        \\&=\int_{\frac{\pi}{2}}^{0}b^{2}\sin ^{2}t \left(-a \sin t \right)\mathrm{d}t
        \\&=ab^{2}\left[\frac{1}{3}\cos ^{3}x -\cos x\right]_{0}^{\frac{\pi}{2}}
        \\&=\frac{2}{3}\pi ab^{2}
      .\end{aligned}
      \]
    \subsection{直角坐标系截面面积已知的情况}
      尝试映射到二维,
      考虑到截面面积本身就是一个从二维映射到一维的函数,
      继承知名球星``第一次压扁成这样的我''的精神,
      将三维体积映射到二维平面即可.
      设截面面积是关于$x$的函数$A\left(x\right)$,
      那么结果为$\int_{a}^{b}A\left(x\right)\mathrm{d}x$.

  \section{弧长}
    \subsection{直角坐标}
      每个函数值对长度的实际贡献为$\sqrt{1+y^{\prime 2}}$,
      对这些贡献进行积分即可.
      例如对于曲线$y=\frac{2}{3}x^{\frac{3}{2}}$,
      可以这样计算:
      \[
      \begin{aligned}
        s&=\int_{a}^{b}\sqrt{1+x}\mathrm{d}x
        \\&=\left[\frac{2}{3}\left(1+x\right)^{\frac{3}{2}}\right]_{a}^{b}
        \\&=\frac{2}{3}\left(\left(1+b\right)^{\frac{3}{2}}-\left(1+a\right)^{\frac{3}{2}}\right)
      .\end{aligned}
      \]
    \subsection{参数方程}
      可以直接建立储存函数的直角参数系到储存积分结果的直角系的映射,
      这样每个函数值对长度的实际贡献为$\sqrt{x^{\prime 2}+y^{\prime 2}}\mathrm{d}t$.
      也可以选择通过换元的方式来思考,
      例如对于任意曲线,
      可以这样计算:
      \[
      \begin{aligned}
        s&=\int_{a}^{b}\sqrt{1+\frac{y^{\prime 2}}{x^{\prime 2}}}\mathrm{d}x
        \\&=\int_{\alpha}^{\beta}x^{\prime}\sqrt{1+\frac{y^{\prime 2}}{x^{\prime 2}}}\mathrm{d}t
        \\&=\int_{\alpha}^{\beta}\sqrt{x^{\prime 2}+y^{\prime 2}}\mathrm{d}t
      .\end{aligned}
      \]
      
      然而, 相比通过换元把参数长度映射到直角长度, 再映射到积分结果,
      一步映射更自然也更好理解.

    \subsection{极坐标}
      稍微画下图就能知道,
      每个函数值对长度的实际贡献为$\sqrt{\rho^{2}+\rho^{\prime 2}}\mathrm{d}\theta$.
      也可以选择通过换元方式推导,
      不过步骤较为繁琐且没有必要, 在这里便不展开了.
      例如对于阿基米德螺线$\rho=a \theta$一圈的长度,
      可以这样计算:
      \[
      \begin{aligned}
        s&=\int_{0}^{2\pi}\sqrt{a^{2}\theta^{2}+a^{2}}\mathrm{d}\theta
        \\&=\frac{a}{2}\left(2\pi\sqrt{1+4\pi^{2}}+\ln\left(2\pi+\sqrt{1+4\pi^{2}}\right)\right)
      .\end{aligned}
      \]
  \section{物理}
    首先考虑变力做功问题:
    由于做功的公式为$W=\int_{a}^{b}F\mathrm{d}s$,
    将距离$s$视为定积分的横坐标即可.
    例如求静电力做功,
    可以这样计算:
    \[
    \begin{aligned}
      W&=\int_{a}^{b}\frac{kq}{r^{2}}\mathrm{d}r
      \\&=kq\left[-\frac{1}{r}\right]_{a}^{b}
      \\&=kq\left(\frac{1}{a}-\frac{1}{b}\right)
    .\end{aligned}
    \]
    
    对于不同种类的力, 套用公式即可.
\end{document}
